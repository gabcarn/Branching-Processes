\setlength{\parindent}{1.5em}
\setlength{\parskip}{0.7em}
\setlength{\headheight}{17.8264pt} 

\usetikzlibrary{graphs, graphdrawing}
\usetikzlibrary{trees}
\newtheorem{definicao_interna}{Definição} 
\newtheorem{teorema}{Teorema}[section]
\pagestyle{fancy}
\fancyhf{}

% Margens da folha
\geometry{
    a4paper,
    left=3cm,
    right=3cm,
    top=3cm,
    bottom=3cm
}

% Informações de Cabeçalho e Rodapé
\lhead{\includegraphics[height=0.4cm]{img/emap.png}} 
\rhead{\textbf{Fundação Getúlio Vargas}}

\lfoot{PIBIC 2025-2026}
\rfoot{Página \thepage\ de \pageref{LastPage}}
\renewcommand{\footrulewidth}{0.4pt}

% Ambientes de teoremas
\newtheorem{theorem}{Teorema}[section]
\newtheorem{proposition}{Proposição}
\newtheorem{corollary}{Corolário}
\newtheorem{lemma}{Lema}

\theoremstyle{definition}
\newtheorem{exercise}{Exercício}[section]
\newtheorem{definition}{Definição}
\newtheorem{example}{Exemplo}[section]
\newtheorem{problem}{Problema}[section]

%Redefinindo comandos 
\DeclareMathOperator{\sen}{\sin} 

\newcommand{\R}{\mathbb{R}}
\newcommand{\N}{\mathbb{N}}
\newcommand{\Z}{\mathbb{Z}}
\newcommand{\Q}{\mathbb{Q}}
\newcommand{\Pp}{\mathbb{P}}
\newcommand{\E}{\mathbb{E}}
\newcommand{\C}{\mathbb{C}}

% Para correções
\newcommand{\aqui}{\textcolor{red}{Corrigir}}
