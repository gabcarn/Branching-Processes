%---------------------------
% CONFIGURAÇÃO DE PÁGINA
%---------------------------
\pagestyle{fancy}
\fancyhf{}
\renewcommand{\headrulewidth}{0pt}

%---------------------------
% AMBIENTES DE TEOREMAS
%---------------------------
\tcbuselibrary{theorems, skins}
\usetikzlibrary{graphs, graphdrawing, trees}

\newtcbtheorem[number within=subsection]{teorema}{Teorema}%
{enhanced,
 arc=0mm,
 outer arc=0mm,
 boxrule=0mm,
 toprule=.5mm,
 bottomrule=0mm,
 left=1mm,
 right=1mm,
 titlerule=0mm,
 toptitle=0mm,
 bottomtitle=1mm,
 top=0mm,
 colframe=blue!20!white,
 colback=blue!5!white,
 coltitle=blue!50!black,
 title style={top color=white,bottom color=white, middle color=white},
 fonttitle=\bfseries\sffamily\normalsize,
 fontupper=\normalsize,
 description delimiters parenthesis,
 separator sign none
}{thm}

\newtcbtheorem[use counter from=teorema]{lema}{Lema}{enhanced,
 arc=0mm, outer arc=0mm, boxrule=0mm, toprule=.5mm, bottomrule=0mm,
 left=1mm, right=1mm, titlerule=0mm, toptitle=0mm, bottomtitle=1mm, top=0mm,
 colframe=blue!20!white, colback=blue!5!white, coltitle=blue!50!black,
 title style={top color=white,bottom color=white, middle color=white},
 fonttitle=\bfseries\sffamily\normalsize, fontupper=\normalsize,
 description delimiters parenthesis, separator sign none
}{lem}

\newtcbtheorem[use counter from=teorema]{proposicao}{Proposição}{enhanced,
 arc=0mm, outer arc=0mm, boxrule=0mm, toprule=.5mm, bottomrule=0mm,
 left=1mm, right=1mm, titlerule=0mm, toptitle=0mm, bottomtitle=1mm, top=0mm,
 colframe=blue!20!white, colback=blue!5!white, coltitle=blue!50!black,
 title style={top color=white,bottom color=white, middle color=white},
 fonttitle=\bfseries\sffamily\normalsize, fontupper=\normalsize,
 description delimiters parenthesis, separator sign none
}{prop}

\newtcbtheorem[use counter from=teorema]{corolario}{Corolário}{enhanced,
 arc=0mm, outer arc=0mm, boxrule=0mm, toprule=.5mm, bottomrule=0mm,
 left=1mm, right=1mm, titlerule=0mm, toptitle=0mm, bottomtitle=1mm, top=0mm,
 colframe=blue!20!white, colback=blue!5!white, coltitle=blue!50!black,
 title style={top color=white,bottom color=white, middle color=white},
 fonttitle=\bfseries\sffamily\normalsize, fontupper=\normalsize,
 description delimiters parenthesis, separator sign none
}{cor}

\newtcbtheorem[use counter from=teorema]{exemplo}{Exemplo}{enhanced,
 arc=0mm, outer arc=0mm, boxrule=0mm, toprule=.5mm, bottomrule=0mm,
 left=1mm, right=1mm, titlerule=0mm, toptitle=0mm, bottomtitle=1mm, top=0mm,
 colframe=green!20!white, colback=green!5!white, coltitle=green!50!black,
 title style={top color=white,bottom color=white, middle color=white},
 fonttitle=\bfseries\sffamily\normalsize, fontupper=\normalsize,
 description delimiters parenthesis, separator sign none
}{ex}

\newtcbtheorem[use counter from=teorema]{definicao}{Definição}{enhanced,
 arc=0mm, outer arc=0mm, boxrule=0mm, toprule=.5mm, bottomrule=0mm,
 left=1mm, right=1mm, titlerule=0mm, toptitle=0mm, bottomtitle=1mm, top=0mm,
 colframe=red!20!white, colback=red!5!white, coltitle=red!50!black,
 title style={top color=white,bottom color=white, middle color=white},
 fonttitle=\bfseries\sffamily\normalsize, fontupper=\normalsize,
 description delimiters parenthesis, separator sign none
}{def}

\newtcbtheorem[use counter from=teorema]{obs}{Observação}{enhanced,
 arc=0mm, outer arc=0mm, boxrule=0mm, toprule=.5mm, bottomrule=0mm,
 left=1mm, right=1mm, titlerule=0mm, toptitle=0mm, bottomtitle=1mm, top=0mm,
 colframe=green!20!white, colback=green!5!white, coltitle=green!50!black,
 title style={top color=white,bottom color=white, middle color=white},
 fonttitle=\bfseries\sffamily\normalsize, fontupper=\normalsize,
 description delimiters parenthesis, separator sign none
}{rem}

%---------------------------
% PROBLEMAS E EXERCÍCIOS
%---------------------------
\newcounter{problema}
\newenvironment{problema}{\refstepcounter{problema}%
\vskip.25\baselineskip%
\noindent\textbf{Problema \theproblema.}\hspace{0.5em}}{\vskip.5\baselineskip}

\newcounter{exercicio}
\newenvironment{exercicio}{\refstepcounter{exercicio}%
\vskip.25\baselineskip%
\noindent\textbf{Exercício \theexercicio.}\hspace{0.5em}}{\vskip.5\baselineskip}

\newcounter{questao}
\NewDocumentEnvironment{questao}{ o }{%
  \refstepcounter{questao}%
  \noindent\textbf{Questão~\thequestao}%
  \IfValueT{#1}{\ (#1 pontos)}\quad
}{\par\medskip}

%---------------------------
% AMBIENTES DE DEMONSTRAÇÃO E SOLUÇÃO
%---------------------------
\makeatletter
\newenvironment{dem}{%
  \par
  \pushQED{\qed}%
  \normalfont \topsep6\p@\@plus6\p@\relax
  \trivlist
  \item[\hskip\labelsep\itshape\color{blue!50!black}Demonstração\@addpunct{.}]\ignorespaces
}{%
  \popQED\endtrivlist\@endpefalse
}

\newenvironment{sol}{%
  \par
  \pushQED{\qed}%
  \normalfont \topsep6\p@\@plus6\p@\relax
  \trivlist
  \item[\hskip\labelsep\itshape\color{green!50!black}Solução\@addpunct{.}]\ignorespaces
}{%
  \popQED\endtrivlist\@endpefalse
}

\newenvironment{solution}{%
  \par
  \pushQED{\qed}%
  \normalfont \topsep6\p@\@plus6\p@\relax
  \trivlist
  \item[\hskip\labelsep\itshape\color{green!50!black}Solution\@addpunct{.}]\ignorespaces
}{%
  \popQED\endtrivlist\@endpefalse
}
\makeatother

%---------------------------
% COMANDOS MATEMÁTICOS
%---------------------------
\DeclareMathOperator{\sen}{\sin} 

\newcommand{\R}{\mathbb{R}}
\newcommand{\N}{\mathbb{N}}
\newcommand{\Z}{\mathbb{Z}}
\newcommand{\Q}{\mathbb{Q}}
\newcommand{\Pp}{\mathbb{P}}
\newcommand{\E}{\mathbb{E}}
\newcommand{\C}{\mathbb{C}}

%---------------------------
% CORREÇÕES
%---------------------------
\newcommand{\aqui}{\textcolor{red}{Corrigir}}