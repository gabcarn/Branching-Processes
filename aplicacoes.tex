\section{Aplicações}

\subsection{Mutação}

Suponha uma população que pode ser modelada através de um modelo de cadeias ramificação, porém cada novo indivíduo
possui uma probabilidade $\gamma$ de possuir uma mutação. Qual a probabilidade de que essa população contraia essa
mutação?

Seja $X$ a variável aleatória que conta todos os nascimentos a partir do primeiro indivíduo, ou seja:

\[
X = \sum_{i \geq 0} Z_i
\]

onde $Z_0 = 1$. Logo a FGP  de $X$ é dada por 

\[
\Pi_{X}(s) = \sum_{k \geq 0 }s^{k}P(X=k) = \E(s^{k})
\]

Condicionando $\Pi_{X}(s)$ no número de indivíduos da primeira geração:

\[
\Pi_{X}(s) = \E(s^{X}) = \sum_{k \geq 0}\E(s^{X} | X_1 = k)P(Z_1 = k | Z_0 = 1) = \E(\E(s^{x} | Z_1))
\]

Tomando $Z_1 = k$, temos $k$ indivíduos no tempo 1, que começam $k$ processos de ramificação independentes, logo $X$
é a soma do indivíduo original somado aos indivíduos de cada um desses $k$ processos de ramificação, que possuem mesma
distribuição de X, ou seja:

$\E(s^{X} | Z_1 = k) = \E(s^{1 + X_1 + \dots + X_k})$

Por independência temos que

\[\E(s^{x} | Z_1 = k ) = s\E(s^{X_1})\dots\E(s^{X_k}) = s\E(s^{X})^k\]

Como a função geradora de probabilidade da distribuição do número de filhos é 

\[
f(s) = \sum p_{k}s^{k}
\]

podemos afirmar que

\[
\Pi_X(s) = s f (\Pi_X(s))
\].

Vamos agora considerar o caso particular onde $p_0 = 1 - p $ e $p_2 = p$. A FGP dessa distribuição é 

\[
f(s) = p_0s^{0} + p_2s^2 = 1 - p + ps^2
\]

então

\[
\Pi_{X}(s) = s f (\Pi_X(s)) = s(1-p + p \Pi_{X}(s)^2) = s - ps + ps\Pi_{X}(s)^2 \implies ps\Pi_{X}(s)^2
- \Pi_{X}(s) + s(1 - p) = 0
\]

provido do fato de que $\lim_{s \to 0)\Pi_{X}(s) = 0$, podemos resolver a equação, encontrando

\[
\Pi_{X}(s) = \frac{1}{2ps} - \sqrt{\frac{1 - 4ps^2(1-p)}{4p^2s^2}}
\]

