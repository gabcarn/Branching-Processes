\section{Aplicações}

\subsection{Mutação}

Suponha uma população que pode ser modelada através de um modelo de cadeias ramificação, porém cada novo indivíduo
possui uma probabilidade $\gamma$ de possuir uma mutação. Qual a probabilidade de que essa população contraia essa
mutação?

Seja $X$ a variável aleatória que conta todos os nascimentos a partir do primeiro indivíduo, ou seja:

\[
X = \sum_{i \geq 0} Z_i
\]

onde $Z_0 = 1$. Logo a FGP  de $X$ é dada por 

\[
\Pi_{X}(s) = \sum_{k \geq 0 }s^{k}P(X=k) = \E(s^{k})
\]

Condicionando $\Pi_{X}(s)$ no número de indivíduos da primeira geração:

\[
\Pi_{X}(s) = \E(s^{X}) = \sum_{k \geq 0}\E(s^{X} | X_1 = k)P(Z_1 = k | Z_0 = 1) = \E(\E(s^{x} | Z_1))
\]

Tomando $Z_1 = k$, temos $k$ indivíduos no tempo 1, que começam $k$ processos de ramificação independentes, logo $X$
é a soma do indivíduo original somado aos indivíduos de cada um desses $k$ processos de ramificação, que possuem mesma
distribuição de X, ou seja:

$\E(s^{X} | Z_1 = k) = \E(s^{1 + X_1 + \dots + X_k})$

Por independência temos que

\[\E(s^{x} | Z_1 = k ) = s\E(s^{X_1})\dots\E(s^{X_k}) = s\E(s^{X})^k\]

Como a função geradora de probabilidade da distribuição do número de filhos é 

\[
f(s) = \sum p_{k}s^{k}
\]

podemos afirmar que

\[
\Pi_X(s) = s f (\Pi_X(s))
\].

Vamos agora considerar o caso particular onde $p_0 = 1 - p $ e $p_2 = p$. A FGP dessa distribuição é 

\[
f(s) = p_0s^{0} + p_2s^2 = 1 - p + ps^2
\]

então

\[
\Pi_{X}(s) = s f (\Pi_X(s)) = s(1-p + p \Pi_{X}(s)^2) = s - ps + ps\Pi_{X}(s)^2 \implies ps\Pi_{X}(s)^2
- \Pi_{X}(s) + s(1 - p) = 0
\]

provido do fato de que $\lim_{s \to 0)\Pi_{X}(s) = 0$, podemos resolver a equação, encontrando

\[
\Pi_{X}(s) = \frac{1}{2ps} - \sqrt{\frac{1 - 4ps^2(1-p)}{4p^2s^2}} = \frac{1}{2ps}(1- \sqrt{1- 4ps^2(1 - p)})
\].

Queremos agora encontrar uma expansão em série de potências para $\Pi_X$. Usando a Fórmula de Newton, temos o
seguinte resultado.

\begin{lemma}{título}{não sei}
    
Seja $c_n = \frac{(2n - 2)}{2^{2n-1}n!(n-1)!} = \frac{1}{n}2^{1-2n}\binom{2n-2}{n-1}, / n \geq 1 $. Então para
$x \in (-1,1)$,

\[
\sum_{n = 1}^{\infty} c_nx^n = - \sum_{n = 1}^{\infty}\binom{1/2}{n}(-x)^n = 
-(\sum_{n=0}^{\infty}\binom{1/2}{n}(-x)^n-1) = 1 - \sqrt{1-x}
\].

\end{lemma}

Voltando para o problema, usando $x = 4ps^2(1-p)$ e $c_n$ como no Lema acima, temos:

\[
\Pi_{X}(s) = \frac{1}{2ps}\sum_{n=1}^{\infty}c_n(4p(1-p)s^2)^n = \sum_{n=1}^{\infty}\frac{(2n-2)!}{n!(n-1)!}p^{n-1}(1-p)^na^{2n-1}
\]

Temos assim uma expansão em série de potências para $g(s), s \in (-1,1)$, o que nos fornece a distribuição de $X$. Para $n \geq 1$,

\[
P(X = 2n -1 | Z_0 = 1) = \frac{(2n-2)!}{n!(n-1)}p^{n-1}(1-p)^n.
\]

Voltando agora ao problema inicial, sejam $M$ o evento de que a mutação aparece na populaçãop e $M^c$ o complemento de $M$.
Se $X$ é infinito, então a probabilidade da população contrair a mutação é sempre 1, logo estamos interessados nos casos em que 
a população é extinta em algum momento. A probabilidade de que a mutação nunca apareça é:

\[
P(M^c) = P(M^c, X < \infty) = \sum_{k = 1}^{\infty} P(M^c | X = k)P(X = k) = \sum_{k = 1}^{\infty}(1 - \gamma)^{k-1}P(X = k)
\].

Porém sabemos que 
\[ 
P(x = 2n -1 | Z_0 = 1 ) = \frac{(2n - 2)!}{k!(n-1)!}p^{n-1}(1-p)^n
\],

logo:

\[
P(M^c) = \sum_{k = 1}^{\infty}(1 - \gamma )^{2k -2}\frac{(2k-2)}{k!(k-1)!}p^{k-1}(1-p)^k
\].

Reorganizando os termos,

\[
P(M^c) = \frac{1}{2p(1 - \gamma)^2}\sum_{k = 1}^{\infty}\frac{(2k-2)!}{k!(k-1)!2^{2k -1}}(2^2(1-\gamma)^2(1-p)p)^k
\],

Assim, pelo Lema já apresentado,

\[
P(M^c) = \frac{1}{2p(1-\gamma)^2}(1 - \sqrt{1- 4(1-p)(1 - \gamma)^2})
\].

Finalmente,

\[
P(M) = 1 -  \frac{1}{2p(1-\gamma)^2}(1 - \sqrt{1- 4(1-p)(1 - \gamma)^2})
\].