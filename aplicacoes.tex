\section{Aplicações}

O estudo de cadeias de ramificação pode ser aplicado em vários cenários. Mostraremos aqui algumas aplicações biológicas
do estudo de cadeias de ramificação, sendo o segundo modelo uma extensão do primeiro.

\subsection{Mutação}

Suponha uma população que pode ser modelada através de um modelo de cadeias ramificação, porém cada novo indivíduo
possui uma probabilidade $\gamma$ de possuir uma mutação. Qual a probabilidade de que essa população contraia essa
mutação?

Seja $X$ a variável aleatória que conta todos os nascimentos a partir do primeiro indivíduo, ou seja:

\[
X = \sum_{i \geq 0} Z_i
\]

onde $Z_0 = 1$. Logo a FGP  de $X$ é dada por 

\[
\Pi_{X}(s) = \sum_{k \geq 0 }s^{k}P(X=k) = \E(s^{k})
\]

Condicionando $\Pi_{X}(s)$ no número de indivíduos da primeira geração:

\[
\Pi_{X}(s) = \E(s^{X}) = \sum_{k \geq 0}\E(s^{X} | X_1 = k)P(Z_1 = k | Z_0 = 1) = \E(\E(s^{x} | Z_1))
\]

Tomando $Z_1 = k$, temos $k$ indivíduos no tempo 1, que começam $k$ processos de ramificação independentes, logo $X$
é a soma do indivíduo original somado aos indivíduos de cada um desses $k$ processos de ramificação, que possuem mesma
distribuição de X, ou seja:

$\E(s^{X} | Z_1 = k) = \E(s^{1 + X_1 + \dots + X_k})$

Por independência temos que

\[\E(s^{x} | Z_1 = k ) = s\E(s^{X_1})\dots\E(s^{X_k}) = s\E(s^{X})^k\]

Como a função geradora de probabilidade da distribuição do número de filhos é 

\[
\phi(s) = \sum p_{k}s^{k}
\]

podemos afirmar que

\[
\Pi_X(s) = s \phi (\Pi_X(s))
\].

Vamos agora considerar o caso particular onde $p_0 = 1 - p $ e $p_2 = p$. A FGP dessa distribuição é 

\[
\phi(s) = p_0s^{0} + p_2s^2 = 1 - p + ps^2
\]

então

\[
\Pi_{X}(s) = s \phi (\Pi_X(s)) = s(1-p + p \Pi_{X}(s)^2) = s - ps + ps\Pi_{X}(s)^2 \implies ps\Pi_{X}(s)^2
- \Pi_{X}(s) + s(1 - p) = 0
\]

provido do fato de que $\lim_{s \to 0)\Pi_{X}(s) = 0$, podemos resolver a equação, encontrando

\[
\Pi_{X}(s) = \frac{1}{2ps} - \sqrt{\frac{1 - 4ps^2(1-p)}{4p^2s^2}} = \frac{1}{2ps}(1- \sqrt{1- 4ps^2(1 - p)}).
\]

Queremos agora encontrar uma expansão em série de potências para $\Pi_X$. Usando a Fórmula de Newton, temos o
seguinte resultado.

\begin{lemma}{Lema}{4.1.1} Seja $c_n = \frac{(2n - 2)}{2^{2n-1}n!(n-1)!} = \frac{1}{n}2^{1-2n}\binom{2n-2}{n-1}, / n \geq 1 $. Então para
$x \in (-1,1)$,

\[
\sum_{n = 1}^{\infty} c_nx^n = - \sum_{n = 1}^{\infty}\binom{1/2}{n}(-x)^n = 
-(\sum_{n=0}^{\infty}\binom{1/2}{n}(-x)^n-1) = 1 - \sqrt{1-x}.
\]

\end{lemma}

Voltando para o problema, usando $x = 4ps^2(1-p)$ e $c_n$ como no Lema acima, temos:

\[
\Pi_{X}(s) = \frac{1}{2ps}\sum_{n=1}^{\infty}c_n(4p(1-p)s^2)^n = \sum_{n=1}^{\infty}\frac{(2n-2)!}{n!(n-1)!}p^{n-1}(1-p)^na^{2n-1}.
\]

Temos assim uma expansão em série de potências para $g(s), s \in (-1,1)$, o que nos fornece a distribuição de $X$. Para $n \geq 1$,

\[
P(X = 2n -1 | Z_0 = 1) = \frac{(2n-2)!}{n!(n-1)}p^{n-1}(1-p)^n.
\]

Voltando agora ao problema inicial, sejam $M$ o evento de que a mutação aparece na populaçãop e $M^c$ o complemento de $M$.
Se $X$ é infinito, então a probabilidade da população contrair a mutação é sempre 1, logo estamos interessados nos casos em que 
a população é extinta em algum momento. A probabilidade de que a mutação nunca apareça é:

\[
P(M^c) = P(M^c, X < \infty) = \sum_{k = 1}^{\infty} P(M^c | X = k)P(X = k) = \sum_{k = 1}^{\infty}(1 - \gamma)^{k-1}P(X = k).
\]

Porém sabemos que 
\[ 
P(x = 2n -1 | Z_0 = 1 ) = \frac{(2n - 2)!}{k!(n-1)!}p^{n-1}(1-p)^n,
\]

logo:

\[
P(M^c) = \sum_{k = 1}^{\infty}(1 - \gamma )^{2k -2}\frac{(2k-2)}{k!(k-1)!}p^{k-1}(1-p)^k.
\]

Reorganizando os termos,

\[
P(M^c) = \frac{1}{2p(1 - \gamma)^2}\sum_{k = 1}^{\infty}\frac{(2k-2)!}{k!(k-1)!2^{2k -1}}(2^2(1-\gamma)^2(1-p)p)^k,
\]

Assim, pelo Lema 4.1.1,

\[
P(M^c) = \frac{1}{2p(1-\gamma)^2}(1 - \sqrt{1- 4(1-p)(1 - \gamma)^2}).
\]

Finalmente,

\[
P(M) = 1 -  \frac{1}{2p(1-\gamma)^2}(1 - \sqrt{1- 4(1-p)(1 - \gamma)^2}).
\]

\subsection{Resistência a um determinado Medicamento}

Imagine agora que a população trabalhada anteriormente é uma população de patógenos, e que a mutação torna cada indivíduo
resistente a um determinado medicamento usado no tratamento contra esse patógeno. Qual a probabilidade de que a população
seja extinta antes do aparecimento dessa resistência?

Continuando nas mesmas condinções da aplicação anterior, vimos que caso a população inicialmente possua apenas 1 indivíduo,
a probabilidade de não ocorrer a mutação é

\[
P(M^c) = \frac{1}{2p(1-\gamma)^2}(1 - \sqrt{1- 4(1-p)(1 - \gamma)^2}).
\]

Porém quando uma pessoa inicia um tratamento para uma doença ela já possui alguns sintomas e consequentemente o número de patógenos
é maior que 1.Tomando $Z_0 = N$, podemos definir a seguinte função

\[
f(N,\gamma,p) = P(M^c | Z_0 = N).
\]

Como cada um dos potógenos possui seu próprio porcesso de ramificação de forma independente dos demais, a probabilidade que queremos
calcular é

\[
f(N,\gamma,p) = f(1,\gamma,p)^N = (\frac{1}{2p(1-\gamma)^2}(1 - \sqrt{1- 4(1-p)(1 - \gamma)^2}))^N
\]

que é uma expressão um pouco complicada de se trabalhar. 

Queremos estudar o comportamento da função para diferentes valores de $p$. Para facilitar um pouco os cálculos iremos utilizar 
aproximação linear para obter uma expressão mais simples.

\begin{definition}
Dada uma função $f(x)$ continua, diferenciável e com variável real x,

\[
f(x) = f(a) + f'(a)(x-a) + o(x)
\]

Onde $o(x)$ é uma função que representa o erro ($\lim_{x \to a} \frac{o(x)}{x} = 0$).Uma aproximação linear de f(x) é obtida desconsiderando
a função erro, isto é,

\[
f(x) \approx f(a) + f'(a)(x-a),
\]

para valores próximos de a, a curva descrita pela função $f(x)$ se aproxima de uma reta.
\end{definition}

Dada a definição, podemos separar nossa função em partes e encontrar uma aproximação linear para cada.

Para a primeira função $f_1(\gamma) = (1 - \gamma)^{-2}$ temos a aproximação linear

\[
f_1(\gamma) \approx 1 + 2\gamma
\]

Para a função $f_2(\gamma) = \sqrt{1 - 4p(1-p)(1-\gamma)^2}$ temos a aproximação linear

\[
f_2(\gamma) \approx = |1 - 2p|(1 + \frac{4p(1-p)}{(1-2p)^2}\gamma).
\]


Usando as aproximações de $f_1$ e $f_2$ conseguimos para $p < \frac{1}{2}$ a seguinte aproximação para nossa expressão

\[
f(N, \gamma, p) \approx (\frac{(1 + 2\gamma)}{2p}(1 - (1 - 2p)(1 + \frac{4p(1-p)}{(1- 2p)^2}\gamma)))^N
\]
\[
= (\frac{(1+2\gamma)}{2p}(2p - \frac{4p(1-p)}{(1-2p)}\gamma))^n
\]
\[
= ((1 + 2\gamma)(1 - \frac{2(1-p)}{(1-2p)}\gamma))^N.
\]

Fazendo agora $f_3(\gamma) = (1 + 2\gamma)(1 - \frac{2(1-p)}{(1-2p)}\gamma)$, encontramos a seguinte aproximação linear

\[
f_3(\gamma) \approx 1 - \frac{2p}{1-2p}\gamma.
\]

Logo,

\[
f(N, \gamma, p) \approx (1 - \frac{2p}{1-2p})^N, \ \text{para} \ p < \frac{1}{2}.
\]

Podemos agora limitar essa aproximação utilizando o fato $e^{-x} > 1-x \forall x > 0$, ou seja,

\[
f(N, \gamma, p) \approx (1- \frac{2p}{1-2p}\gamma)^N < exp(-\frac{2p}{1-2p}N\gamma).
\]

A partir disso, observa-se que para $p < \frac{1}{2}$ fixo, a probabilidade de não ter resistência depende do parâmetro 
$m \equiv N\gamma$, e decresce exponenciamente quando $m$ aumenta.

Usando agora as aproximações de $f_1$ e $f_2$ e $p > \frac{1}{2}$ fixo na expressão original, obtemos a seguinte aproximação para a expressão

\[
f(N, \gamma, p) \approx ((1 + 2\gamma)\frac{(1-p)}{p}(1 + \frac{2p}{(1 - 2p)}\gamma))^N.
\]

Usando então $f_4(\gamma) = (1 + 2\gamma)\frac{(1-p)}{p}(1 + \frac{2p}{(1 - 2p)}\gamma)$, obtemos a aproximação linear

\[
f_4(\gamma) \approx \frac{(1-p)}{p}(1 - \frac{2(1-p)}{(2p -1)}\gamma).
\]

Encontramos então o seguinte resultado,

\[
f(N, \gamma, p) \approx (\frac{(1-p)}{p}(1 - \frac{2(1-p)}{(2p-1)}\gamma))^N, \ \text{para} \ p > \frac{1}{2}.
\]

Usando novamente o fato $e^{-x} > 1 - x \forall x > 0$, obtemos um limitante para essa aproximação,

\[
f(N, \gamma, p) \approx (\frac{(1-p)}{p}(1 - \frac{2(1-p)}{(2p-1)}\gamma))^N < (\frac{1-p}{p})^N exp(-\frac{2(1-p)}{(2p-1)}N\gamma),
\]

e como consequência temos também que 

\[
f(N, \gamma, p) < (\frac{1-p}{p})^n.
\]

mostrando que a medida que $N$ aumenta, a probabilidade de que os patógenos adquiram a resistência ao medicamento também aumenta, tornando 
o tratamento bem mais complicado.