\documentclass[xcolor=dvipsnames,t,aspectratio=169]{beamer} %t para ficar alinhado no topo do slide

\usecolortheme{rose}
\usecolortheme{dolphin}
\usetheme{Boadilla}

\input{imports}

\input{settings}

\input{commands}

\titlegraphic{
    \includegraphics[scale = 0.25]{emap_logo}
}

\logo{
\begin{tikzpicture}[overlay,remember picture]
\node[left=1.1cm, below=0.2cm] at (current page.30){
    \includegraphics[width=0.165\textwidth]{emap_logo}};
\end{tikzpicture}
}

\lstset{style=mystyle}

\newcommand{\highlight}[1]{{\color{fgv_light_blue} #1}}

\title{Cadeias de Ramificação} 


\author{
  Gabriel Carneiro Nunes da Silva
}


\date{{\color{fgv_dark_blue}  \textbf{Escola de Matemática Aplicada}\\ Dezembro de 2025 }}

\begin{document}

\frame[plain]{\titlepage}
\setcounter{framenumber}{0}

\begin{frame}[c]{Definição}

    \begin{definicao}{Processo de Ramificação}{br-processes}
    Um \textbf{processo de ramificação} é definido da seguinte forma:

        \begin{itemize}
            \item Um único indivíduo no instante $n = 0$.
            \item Cada indivíduo vive exatamente uma unidade de tempo, depois produz $X$ descendentes e morre.
            \item O número de descendentes $X$ assume valores $0, 1, 2, \ldots$, e a probabilidade de produzir $k$ descendentes é $\mathbb{P}(X = k) = p_k$.
            \item Todos os indivíduos se reproduzem independentemente. Os indivíduos $1, 2, \ldots, n$ têm tamanhos de família $X_1, X_2, \ldots, X_n$, onde cada $X_i$ tem a mesma distribuição que $X$.
            \item Seja $Z_n$ o \textit{número de indivíduos nascidos} no instante $n$, para $n = 0, 1, 2, \ldots$.
            Interprete $Z_n$ como o \textit{``tamanho'' da geração $n$}.
            \item Então o processo de ramificação é $\{Z_0, Z_1, Z_2, Z_3, \ldots\} = \{Z_n : n \in \mathbb{N}\}$.
        \end{itemize}
    \end{definicao}
\end{frame}

\begin{frame}[c]{Ilustração}
    \centering
    \includegraphics[width=0.8\textwidth]{cadeia.png}
\end{frame}

\begin{frame}[c]{Sim}
    \begin{lembre}{}{}
        Outra forma de analisar, é observando que
        \[
            Z_{n+1} = \sum_{i=1}^{Z_n} X^{(n)}_i
        \]
    \end{lembre}

\end{frame}

\begin{frame}[c]{Sim}
    \begin{lema}{Valor Esperado e variância de $Z_n$}{lem:espvar}
        Seja $(Z_n)$ uma cadeia de ramificação. Seja $X$ o número de descendentes de um determinado indivíduo (lembre que $X \sim p$), e suponha que $\mathbb{E}[X] = \mu$, e $\mathrm{Var}(X) = \sigma^2$. Então

        \begin{itemize}
            \item $\mathbb{E}(Z_n) = \mu^{n}$
            \item \[
            \mathrm{Var}(Z_n) = 
            \begin{cases}
                \sigma^2 n, & \text{se } \mu = 1, \\[6pt]
                \sigma^2 \mu^{\,n-1}\displaystyle\left(\frac{1-\mu^n}{1-\mu}\right), 
                & \text{se } \mu \neq 1.
            \end{cases}
            \]
        \end{itemize}

    \end{lema}
\end{frame}

\begin{frame}[c]{Sim}
    \begin{proof}[Lema \ref{lem:espvar}]
                De fato, podemos simplesmente fazer as contas:

        \[\bE(Z_n) = \bE\left(\sum_{i=1}^{Z_{n-1}} X_i^{(n)}\right) = \bE\left(\bE\left(\sum_{i=1}^{Z_{n-1}}X_i^{(n)}\,\middle|\, Z_{n-1}\right)\right) = \bE(Z_{n-1} \cdot \bE(X_i^{(n)}))  \]

        \[ = \bE(Z_{n-1}) \cdot \bE(X) = \dots = \bE(Z_0) \cdot \bE(X)^{n} = \bE(X)^{n}\]

        Portanto, $\bE(Z_n) = \mu^{n}$, como queríamos
    \end{proof}
\end{frame}

\begin{frame}[c]{SIm}
    \begin{ex}{}{}
        Considere uma cadeia de ramificação $(Z_n)$ com $p = (\frac{1}{2}, \frac{1}{4}, \frac{1}{8}, \dots)$.

    \end{ex}
    \[
        \bE(X) = \sum_{i=0}^{\infty} i \cdot \mathbb{P}(X = i) = \sum_{i=0}^{\infty} \dfrac{i}{2^{i+1}} 
    \] 

    E portanto, podemos escrever como 

    \[
    \begin{aligned}
    \bE(X) = 
     \tfrac{1}{2} + \tfrac{1}{4} &+ \tfrac{1}{8} + \tfrac{1}{16} + \cdots \\[4pt]
    + \tfrac{1}{4} &+ \tfrac{1}{8} + \tfrac{1}{16} + \cdots  \\[4pt]
    &+  \tfrac{1}{8} + \tfrac{1}{16} + \cdots
    \end{aligned}
    \]
    Logo, realizando a soma dessas progressões geométricas, temos $\bE(X) = 1$, e também $\bE(Z_n) = 1^n = 1$ para todo $n$.

\end{frame}

\begin{frame}[c]{Probabilidade de Extinção}
    Qual a probabibilidade de extinção da cadeia?

    \begin{definicao}{}{}
    Definiremos a probabilidade de extinção como segue:
        \[
        p_e = \Pp(\text{Extinção}) = \Pp\left(\bigcup^{\infty}_{n=0} Z_n = 0\right) = \lim_{n \to \infty} \Pp(Z_n = 0) 
        \]
    \end{definicao}
\end{frame}

\begin{frame}[c]{Exemplos}
    \begin{ex}{}{}
    Abaixo segue alguns exemplos simples
    \begin{enumerate}
        \item População tal que $p_0 = 1$, então na próxima geração certamente ela irá se extinguir, isto é $Z_n = 0 \,\, \forall n \geq 1$, daí, sua probabilidade de extinção é $1$.
        

        \begin{tikzpicture}[
            node distance=1.5cm,
            ball/.style={circle, fill=gray, minimum size=5pt, inner sep=0pt},
            label/.style={above=2pt}
        ]

        \node[ball] (n0) at (0,0) {};

        % X vermelho no meio
        \node[red] (x) [right=of n0] {\Large $\mathbf{\times}$};

        % Labels
        \node[label] at (n0.north) {$X_1^{(0)}$};

        % Linha conectando
        \draw (n0) -- (x);

        \end{tikzpicture}
            
        \item População tal que $p_0 = \frac{1}{100}$ e $p_1 = \frac{99}{100}$, então, no longo prazo, ela irá se extinguir, isto é, existirá um tempo $n_0$ tal que $Z_n = 0 \,\, \forall n \geq n_0$. Daí, sua probabilidade de extinção também é $1$.
        
        \begin{tikzpicture}[
            node distance=1.5cm,
            ball/.style={circle, fill=gray, minimum size=5pt, inner sep=0pt},
            label/.style={above=1pt}
        ]

        % Nós (bolinhas) em linha horizontal
        \node[ball] (n0) at (0,0) {};
        \node[ball] (n1) [right=of n0] {};
        \node[ball] (n2) [right=of n1] {};
        \node        (dots) [right=of n2] {$\cdots$};
        \node        (end) [right=0.5cm of dots] {\textcolor{red}{\Large $\mathbf{\times}$}};


        % Ligações
        \draw (n0) -- (n1);
        \draw (n1) -- (n2);
        \draw (n2) -- (dots);

        % Labels
        \node[label] at (n0.north) {$X_1^{(0)}$};
        \node[label] at (n1.north) {$X_1^{(1)}$};
        \node[label] at (n2.north) {$X_1^{(2)}$};

        \end{tikzpicture}
    \end{enumerate}
    \end{ex}
\end{frame}

\end{document} 