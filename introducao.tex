\section{Introdução}

O estudo da dinâmica das populações é frutífero para a matemática, como sabemos do estudo das equações diferenciais de Verhulst ou dos modelos clássicos de predador-presa, como o de Lotka-Volterra. Nesse trabalho, apresentamos uma teoria que também se originou no estudo da dinâmica populacional, mas que se baseia principalmente na modelagem da reprodução como sendo um processo estocástico: A teoria das Cadeias de Ramificação.

A ideia de utilizar um processo estocástico para estudar o crescimento populacional tem origem nos trabalhos de Francis Galton\footnote{Curiosidade: era também primo de Charles Darwin e foi o idealizador da eugenia} e de Henry William Watson, que estavam interessados na possível extinção de sobrenomes de aristocratas no Reino Unido. Uma cadeia de ramificação, portanto, pode ser descrita da seguinte maneira:

Um indivíduo de uma certa espécie se reproduz (e morre logo em seguida) de maneira assexuada, com o vetor $\textbf{p} = (p_1,p_2 \dots,p_k,\dots)$ sendo a distribuição de probabilidade para o tamanho de sua prole. Queremos modelar a quantidade de indivíduos $Z_n$ na enésima geração, considerando que todos indivíduos de uma mesma geração se reproduzem e morrem num mesmo instante.

Estabelecida a teoria básica, podemos nos fazer as seguintes perguntas típicas:

\begin{itemize}
	\item Qual o número esperado de filhos de cada indivíduo?
	\item Qual a esperança e a variância de $Z_n$?;
	\item É possível achar a distribuição de $Z_n$?;
	\item É possível achar a probabilidade de extinção da cadeia? (ou seja, um instante $N$ no qual $Z_N = 0$);
	\item É possível achar as condições para eventual extinção da cadeia?,
	\item Se é certa a extinção, podemos achar a distribuição do tempo até ela?
\end{itemize}

Planejamos, nas seguintes seções, desenvolver a teoria necessária para tentar responder as perguntas acima. Entretanto, vamos tentar aplicar os nossos conhecimentos já estabelecidos de probabilidade em alguns exemplos motivadores.

\subsection{Exemplos Motivadores}

\begin{exemplo}{}{}
Tome $\textbf{p} \sim k \cdot \text{Bernoulli}(p)$, ou seja, cada indivíduo dessa espécie tem $k$ filhos com probabilidade $p$ ou não tem nenhum filho com probabilidade $1-p$.
\end{exemplo}

No exemplo 1.1.1, algumas das perguntas feitas inicialmente são relativamente fáceis de responder:

\begin{itemize}
	\item O número esperado de filhos é $\mathbb{k\textbf{p}} = k \cdot p$
	\item A esperança de $Z_n$ é


\end{itemize}
