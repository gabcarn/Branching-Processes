\section{Introdução}

Aqui será introduzido

Considere 

$X_n := \text{número de indivíduos no tempo } n$

cada célula $Z$ tem uma probabilidade de ter $k$ filhos denotada por $p_k$

Probabilidade de extinção

\begin{exemplo}{}{}
    \begin{enumerate}
        \item Se eu tenho uma população tal que $p_0 = 1$, então na próxima geração certamente ela irá se extinguir, isto é $X_t = 0 \,\, \forall t \geq 0$, e portanto, sua probabilidade de extinção é $1$.
        \item Se eu te tenho uma população tal que $p_0 = \frac{1}{100}$ e $p_1 = \frac{99}{100}$, então na próxima geração certamente ela irá se extinguir, isto é $X_t = 0$, existirá um tempo $t_0$ tal que $X_t = \,\, \forall t \geq t_0$. Daí, sua probabilidade de extinção também é $1$.
        \item Caso seja uma população tal que $p_0 = \frac{1}{4}$, $p_1 = \frac{1}{2}$. $p_2 = \frac{1}{4}$, mostraremos também que a população irá se extinguir. 
    \end{enumerate}
\end{exemplo}

Daí surge a pergunta natural, como estudar a probabibilidade de extinção? Usaremos a seguinte notação

\[
\Pp_e = \Pp(\text{Extinção}) = \Pp\left(\bigcup^{\infty}_{n=0} X_n = 0\right) = \lim_{n \to \infty} \Pp(X_n = 0) 
\]


\begin{definicao}{Função Geradora de Probabilidade}{}
Considere $X$ uma variável aleatória, definiremos a \textbf{Função Geradora de Probabilidade} $\Pi_X: \R \to \R$ como sendo
    \[
    \Pi_X(S) = \E(S^X) = \sum_{x=0}^{\infty}\Pp(X=x)S^x
    \]
\end{definicao}

Propriedades da função geradora de probabibilidade (fácil verificação)

\begin{enumerate}
    \item $\Pi_X (0) = \Pp(X = 0)$
    \item $\Pi_X'(S) =  \Pp(X = 1)$, de modo geral, $\Pi_X'(0) =  n!\cdot\Pp(X = n)$
    \item $\Pi_X(1) = \sum_{x=0}^{\infty}\Pp(X=x) = 1$
\end{enumerate}

\begin{exemplo}{}{}
    Considere a v.a. $X \sim Bernoulli(p)$, iremos calcular a F.G.P desta variável aleatória. 
    \[
    \Pi_X(S) = \E(S^X) = \Pp(X=0) + \Pp(X = 1)\cdot S = (1-p) + p\cdot S
    \]
\end{exemplo}
