\section{Tempo de Extinção}

Uma das perguntas feitas na introdução foi sobre a distribuição do tempo de extinção de uma cadeia de ramificação. Para ser mais específicos, nos referimos à variável aleatória

\[
\tau = \min \{ n \in \mathbb{Z} : Z_n = 0 \}
\] 

dessa forma, $\tau = t \iff Z_t = 0, Z_{t-1} >0$. É útil achar a distribuição dessa variável porque ela pode ser útil quando a extinção é algo quisto pelo modelo, por exemplo, quando a cadeia de ramificação representa o número de indivíduos infectados numa população. Pela equivalência acima, sabemos que \[\Pp(\tau = t) = \Pp(Z_t = 0, Z_{t-1} > 0)\]. E pela lei da probabilidade total, temos \[\Pp(Z_t = 0, Z_{t-1} >0) = \Pp(Z_t = 0) - \Pp(Z_t = 0, Z_{t-1} =0)\]

No lado direito da igualdade acima, sabemos que $\Pp(Z_t = 0, Z_{t-1})$ é redundante e igual a $\Pp(Z_{t-1} = 0)$. Pela proposição \textbf{3.1.1}, temos que $\Pp(Z_t = 0) = \Pi_{Z_t}(0) = p_{t,0}$. Assim, concluímos que a distribuição do tempo de extinção obedece à equação:

\[\Pp(\tau = t) = p_{t,0} - p_{t-1,0}\]

Com a condição inicial de que $\Pp(\tau = 1) = p_0$. 

Num geral, como já visto, achar os valores de $p_{t,0}$ pode ser uma tarefa complicada, principalmente porque envolve calcular órbitas da $p_0$ pela iteração da função $\Pi_X(s)$ consigo mesma, que pode não ser uma série simples de calcular explicitamente, mas ainda pode ser aproximada pelo computador.

Um outro problema é calcular os momentos de $\tau$. O principal resultado sobre tal assunto é o seguinte:

\begin{proposicao}{}{}


\[\mathbb{E}(\tau) = \begin{cases} 
	\text{ finita }, \text{ se $\mathbb{E}(Z_1)$ < 1 }; \\
	\text{ infinita }, \text{ se $\mathbb{E}(Z_1)$ = 1 e Var($Z_1$) é finita}; \\
	\text{ infinita }, \text{ se $\mathbb{E}(Z_1)$ > 1}; \\
		\end{cases}
\]
\end{proposicao}

A demonstração do afirmado exige alguns conhecimentos sobre o comportamento assintótico de $Z_n$, resultados um pouco mais técnicos que decidimos não incluir no texto.

Dessa forma, concluímos que apesar de conseguirmos simular de maneira mais precisa a distribuição de $\tau$, ainda é muito difícil encontrar uma fórmula fechada para tal.
